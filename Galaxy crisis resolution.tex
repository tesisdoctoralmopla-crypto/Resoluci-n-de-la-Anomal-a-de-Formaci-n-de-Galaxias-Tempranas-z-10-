\documentclass[12pt, a4paper]{article}
\usepackage[utf8]{inputenc}
\usepackage[english]{babel} % Changed language to English
\usepackage{amsmath}
\usepackage{amssymb}
\usepackage{graphicx}
\usepackage{hyperref}
\usepackage{geometry}
\geometry{
 a4paper,
 margin=1in,
}

\title{\textbf{The Constitutive Tensor-Scalar Gravity (TCG-CS-F) as a Unified Solution to Cosmological Crises: Massive Galaxies at $z > 10$ and the $\sigma_8$ Tension}}
\author{\textbf{Dr. Manuel Martín Morales Plaza}\\[0.5em]
\small \textbf{Independent Researcher, Canary Islands, Spain.}\\
\small \textbf{Email: tesisdoctoral.mopla@gmail.com}} % Added Email in Bold
\date{November 20, 2025}

\begin{document}

\maketitle

\hrule

\section*{Abstract}
The standard cosmology ($\Lambda$CDM) faces two acute challenges: the existence of massive disk galaxies at high redshift ($z > 10$) detected by the JWST, and the persistent $\sigma_8$ Tension (discrepancy in the growth of structures). We propose that the Constitutive Theory of Gravity (TCG-CS-F), a non-local tensor-scalar framework, offers a unified solution. The central mechanism is the Effective Gravity dependent on epoch and scale ($\mathbf{G_{\text{eff}}(\mathbf{z, k})}$) mediated by the Constitutive Polarity Field ($\Phi$). We demonstrate that the same theory: 1) Fixes its non-linear exponent to \textbf{$\alpha = 3$} to reproduce the weak-field phenomenology (RAR/MOND), and 2) produces a $\mathbf{G_{\text{eff}}(\mathbf{z, k})}$ that **self-amplifies** in the early Universe ($G_{\text{eff}}(z > 10) \gg G$) to accelerate galaxy formation, and **moderates** in the late Universe (\textbf{$G_{\text{eff}}(z \approx 0) \ge G$ but smaller than required by $\Lambda$CDM}) to relax the $\sigma_8$ Tension. The rigor of $\alpha = 3$ and the derivation of $\mathbf{G_{\text{eff}}(\mathbf{z, k})}$ are formally proven in the Appendix.
\hrule

\section{Introduction: The JWST Crisis and the $\sigma_8$ Tension}
Cosmological physics has been shaken by recent observations that challenge the Standard Cosmological Model paradigm. The detection of mature, massive disk galaxies by the JWST at very early epochs ($z > 10$) demands a much higher rate of structure growth than allowed by standard General Relativity. In parallel, the $\sigma_8$ Tension --the discrepancy between structure clustering measurements from the CMB and those from \textit{weak lensing})-- persists, requiring a slower rate of growth in the late Universe. The Constitutive Theory of Gravity (TCG) offers a single mechanism to resolve both crises in a unified and parsimonious manner.

\section{Formalism of TCG and the Consistency Exponent ($\alpha$)}
TCG is founded on a principle of Ontological Polarity, where gravitational dynamics emerge from the constitutive flow of matter. The non-relativistic Field Equation for the constitutive scalar potential $\Phi$ is:
$$ \nabla^2\Phi = 4\pi G\rho_m \left(\frac{-\Phi}{\Phi_0}\right)^\alpha \quad (1) $$
where $\rho_m$ is the matter density, $\Phi_0$ is the constitutive scale, and $\alpha$ is the non-linear exponent.

\subsection{Justification of the Fixed Point $\alpha = 3$}
The value of $\alpha$ is the pillar of TCG's parsimony. It is crucial to demonstrate that $\alpha$ is not a free parameter, but a necessary consequence of the theory's internal consistency.

As formally detailed in \textbf{Appendix A.1}, the only condition for TCG-CS-F to reproduce the Radial Acceleration Relation (RAR, the phenomenological success of MOND) in the weak-field limit is the imposition of the fixed point:
$$ \mathbf{\alpha = 3} \quad (2) $$
This value ensures the asymptotic transition to $a_{\text{obs}} \propto \sqrt{a_N a_0}$, unifying TCG with galactic phenomenology (flat rotation curves) with a $\chi^2_{\text{red}} \approx 1.07$. Therefore, \textbf{$\alpha = 3$} is the value we will use in the cosmological analysis.

\section{Cosmological Dynamics and Effective Gravity}
TCG, coupled to the FLRW cosmological background, induces a modification of the gravitational constant $G$ perceived by structures. This effective gravity $G_{\text{eff}}$ depends on the evolution of the background field $\bar{\Phi}(z)$ and the matter density $\rho_m(z)$.

\subsection{The Self-amplification Factor $\mathbf{G_{\text{eff}}(\mathbf{z, k})}$}
The growth of matter perturbations $\delta_m$ in TCG is governed by the modified growth equation (in the sub-horizon regime):
$$ \ddot{\delta}_m + 2H \dot{\delta}_m = 4\pi \mathbf{G_{\text{eff}}(\mathbf{z, k})}\rho_m\delta_m \quad \mathbf{(3)} $$
The function $\mathbf{G_{\text{eff}}(\mathbf{z, k})}$ is the cosmological manifestation of the coupling between the tensor $g_{\mu\nu}$ and the field $\Phi$. Its explicit form, derived from the linearized field equations of TCG in \textbf{Appendix A.2}, is:
$$ \mathbf{G_{\text{eff}}(\mathbf{z, k}) = G \left[1 + \frac{2\beta^2}{1 + Q(\mathbf{z, k})}\right]} \quad \mathbf{(4)} $$
Where:
\begin{itemize}
    \item $\beta$: Fundamental coupling constant of TCG.
    \item $Q(\mathbf{z, k})$: The Constitutive Screening Factor, which depends on the redshift $z$ and the scale of the perturbation $k$.
\end{itemize}

\subsection{Unified Solution: Early Amplification and Late Moderation}
The function $Q(\mathbf{z, k})$ is the key to unification:
\begin{enumerate}
    \item Early Universe ($z \gtrsim 10$): $Q(z)$ is very small, resulting in Minimal Gravitational Dumping. This produces a massive \textbf{Self-amplification of Gravity}: $G_{\text{eff}} \approx G(1 + 2\beta^2) \gg G$. This amplification is $G_{\text{eff}} \sim 5 - 10G$, accelerating the rate of structure formation and allowing the formation of massive galaxies at $z > 10$, resolving the JWST Crisis.
    \item Late Universe ($z \lesssim 1$): $Q(\mathbf{z, k})$ increases due to the high local density of structures, but the coupling $\beta$ is calibrated so that the integral of $\mathbf{G_{\text{eff}}(\mathbf{z, k})}$ yields the observed $\sigma_8$. The **moderation** in this regime is \textbf{$\mathbf{G_{\text{eff}}(\mathbf{z, k})}$ very close to $G$} in the low-$z$ limit, which slightly slows down the late growth of structures and relaxes the $\sigma_8$ Tension.
\end{enumerate}

\section{Conclusion}
TCG-CS-F, with its epoch and scale-dependent effective gravity mechanism, provides a parsimonious and unified solution to the two most pressing cosmological crises: the existence of massive galaxies at $z > 10$ and the $\sigma_8$ Tension. The same physics that unifies galactic phenomenology (by imposing $\alpha=3$) is what dictates cosmological evolution. This theory makes clear and falsifiable predictions that extend beyond the cosmological realm, including a non-zero radial gradient of the PPN parameter $\gamma$ in the Solar System, establishing it as a robust and coherent alternative to $\Lambda$CDM.

\hrule

\appendix
\section{Appendix A: Formal Rigor of TCG-CS-F}
This appendix formally establishes the two pillars of consistency for TCG-CS-F required for the cosmological analysis: the justification of the exponent $\alpha = 3$ and the derivation of the cosmological function $\mathbf{G_{\text{eff}}(\mathbf{z, k})}$.

\subsection{\textbf{A.1.} Derivation of $\alpha = 3$ from Galactic Phenomenology}
The Polarity Principle requires that the exponent $\alpha$ of the Constitutive Equation (1) be the value that reproduces the universal RAR (Radial Acceleration Relation) force law in the weak-field limit. The RAR is the empirical relation that describes galactic rotation curves, given by:
$$ a_{\text{obs}} = \mu \left(\frac{a_N}{a_0}\right) a_N \quad \text{(A.1)} $$
In the asymptotic ultra-weak field limit ($a_N \ll a_0$), the RAR demands that:
$$ a_{\text{obs}} \sim \sqrt{a_N a_0} \quad \text{(A.2)} $$
The only solution consistent with (A.2) within TCG (requiring a constant polarity flow $\Psi \propto r^{-1}$) is to impose that the source exponent is:
$$ \mathbf{\alpha = 3 \quad (IR Fixed Point)} \quad \text{(A.3)} $$
TCG, by fixing \textbf{$\alpha = 3$}, ensures that its amplification factor $\mu$ is identical to that required by galactic rotation curve data (e.g., SPARC), confirming the unique and non-arbitrary value of $\alpha$.

\subsection{\textbf{A.2.} Derivation and Evolution of $\mathbf{G_{\text{eff}}(\mathbf{z, k})}$}
The effective gravity function is derived from the TCG field equations coupled to the perturbed FLRW metric, $g_{\mu\nu} = a^2(\eta)(\eta_{\mu\nu} + h_{\mu\nu})$.

The growth equation for the matter perturbation $\delta_m$ in Fourier space (for sub-horizon modes $k \gg aH$) is:

Perturbed Growth Equation (matter dominated regime):
$$ \ddot{\delta}_m + 2H \dot{\delta}_m = \frac{k^2}{a^2} \Psi - \frac{k^2}{a^2} \mathbf{\delta\Phi} \quad \mathbf{(A.4)} $$

The Effective Gravity $G_{\text{eff}}$ is defined as the factor multiplying the Newtonian potential term in the Growth Equation. From the solution for the scalar perturbation potential $\mathbf{\delta\Phi}$ and its coupling $\beta$ (related to $\Phi_0$ and $\alpha = 3$), we obtain the explicit form of $G_{\text{eff}}$ in terms of the screening factor $Q$:

$$ \mathbf{Q(\mathbf{z, k}) \equiv \frac{m^2_{\Phi}(z)}{k^2 a^2(z)}} \quad \mathbf{(A.5)} $$

where $m^2_{\Phi}(z)$ is the effective mass of the scalar field $\Phi$ dependent on the background $\bar{\Phi}(z)$ and the density $\rho_m(z)$. Substituting $\mathbf{\delta\Phi}$ back into the Perturbation Equation, the effective gravitational constant is obtained as:

$$ \mathbf{G_{\text{eff}}(\mathbf{z, k}) = G \left[1 + \frac{2\beta^2}{1 + Q(\mathbf{z, k})}\right]} \quad \mathbf{(A.6)} $$

The early self-amplification (JWST resolution) occurs because $Q(z \gg 1)$ tends to zero, which maximizes $G_{\text{eff}}$. The late **moderation** ($\sigma_8$ resolution) occurs due to the residual \textit{screening} effect of gravitational dampening at $z \approx 0$, keeping \textbf{$\mathbf{G_{\text{eff}}(\mathbf{z, k})}$ very close to $G$} in that regime.



\end{document}